\begin{conclusions}

Este trabajo presenta un sistema de optimización de dos fases para resolver problemas de clasificación arbitrarios de forma justa.
Los principales objetivos de este sistema son (1) sacar provecho de todas las soluciones que fueron producidas mientras se resuelve un problema de \emph{AutoML}, y (2) utilizar estos modelos para construir una solución que además de efectiva sea justa en la toma de decisiones.
La evaluación en los conjuntos de datos \emph{Adult} y de la tarea \emph{HAHA 2019} probaron que el utilizando la estrategia de diversificación \texttt{double-fault} y el método de optimización aquí propuesto, se pueden construir modelos que tengan un buen rendimiento a la vez que mantienen un alto nivel de equidad.
Luego, nuestro trabajo confirma la hipótesis de que una solución efectiva y justa puede ser construida a partir de ensamblar de forma inteligente un subconjunto de los modelos generados mientras se resuelve el problema de \emph{AutoML}.

Dos métricas de diversidad tomadas de la literatura fueron estudiadas: la métrica de \texttt{disagreement} y \texttt{double-fault}.
Para cuantificar la calidad de los resultados obtenidos a partir de utilizar estas métricas, dos métricas adicionales, basadas en el concepto de \emph{ensembles oráculo}, son presentadas en este trabajo.
Los resultados permiten observar como el proceso de selección de modelos base influencia el rendimiento de las técnicas de ensemble.
La métrica de \emph{disagreement} asegura el máximo cubrimiento de los datos, sin embargo, las colecciones construidas utilizando esta estrategia no necesariamente proveen salidas consistentes.
La métrica de \emph{double-fault} brinda los mejores resultados en general.

Una modificación al algoritmo de búsqueda de \emph{AutoGOAL} fue propuesta para aceptar múltiples funciones objetivos.
La capacidad de este para optimizar métricas de equidad y precisión simultáneamente fue evaluada en el conjunto de datos \emph{Adult} y los resultados comparados con otros métodos de la literatura.
Como era de esperarse, nuestra propuesta mostró ser sumamente competitiva y obtener resultados satisfactorios.
En particular este sistema mostró tener la capacidad no solo de encontrar resultados que cumplan con determinadas restricciones de equidad, sino de encontrar diferentes balances entre equidad y precisión.

Los resultados de este trabajo son resumidos a continuación:

\begin{itemize}
    \item Al utilizar técnicas de \emph{AutoML} combinadas con métodos de ensemble y algoritmos multiobjetivo se logra un sistema que de forma agnóstica al modelo logra resolver problemas de clasificación arbitrarios de manera justa.
    \item Se mostró la influencia de las diferentes métricas de diversidad en la formación de modelos base que den lugar a ensembles mas robustos.
    \item Al utilizar un método multiobjetivo para ensamblar un conjunto clasificadores base se logra encontrar modelos \textbf{justos} con alto rendimiento.
    \item Este método permite además trabaja con múltiples métricas de equidad simultáneamente.
\end{itemize}

\end{conclusions}