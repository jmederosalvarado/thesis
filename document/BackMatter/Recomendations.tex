\begin{recomendations}

Este trabajo orientado a la resolución de problemas de clasificación arbitrarios de forma justa y automatizando la mayor parte del proceso.
Sin embargo, una de las limitaciones del sistema propuesto es que requiere que el usuario manualmente indique los atributos protegidos en los datos.
Por lo que se propone integrar el sistema aquí propuesto con propuestas actualmente en desarrollo para la anotación automática de los atributos protegidos de un conjunto de datos.

Actualmente la primera fase de nuestro sistema utiliza un estrategia golosa para computar la diversidad entre los clasificadores y de esta forma seleccionar los modelos base que serán ensamblados.
Este método sin embargo puede ser subóptimo, se recomienda el estudio de otras estrategias, como métodos de \emph{clustering} como alternativa.
La experimentación realizada en este trabajo no estudió el comportamiento del sistema utilizando métodos de aprendizaje profundo.
Se sugiere la exploración en futuros trabajos de como estas técnicas podrían afectar el rendimiento del sistema.
Finalmente la experimentación en conjuntos de datos con características diferentes a los aquí empleados es imprescindible para corroborar el comportamiento del sistema en problemas de mayor escala.

\end{recomendations}
