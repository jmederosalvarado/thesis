\begin{opinion}

En la actualidad, los algoritmos de aprendizaje automático están siendo aplicados en disímiles áreas de la vida humana. En particular, su incorporación a tareas de toma de decisiones de alto riesgo ha dirigido la atención de muchos investigadores hacia una nueva interrogante: ¿estarán siendo “justos” los algoritmos de aprendizaje automático al tomar sus decisiones? El concepto de justicia o equidad se interpreta en este contexto como la ausencia de cualquier prejuicio o favoritismo hacia un individuo o grupo basado en sus características inherentes o adquiridas. El peligro fundamental de ignorar la interrogante planteada anteriormente radica en que los métodos de aprendizaje automático podrían no solo reflejar los sesgos presentes en nuestra sociedad, sino que también podrían amplificarlos. Resolver problemas de forma justa debería convertirse en un estándar en todos los contextos en los que es aplicable. En este marco se desarrolla la tesis de licenciatura de Jorge Mederos Alvarado, con quien pude trabajar este último año en el diseño y validación de un sistema para la resolución de problemas de clasificación de forma justa.

La propuesta de Jorge consiste en un enfoque que combina técnicas de AutoML, métodos de ensemble y optimización multiobjetivo, resultando en un sistema que permite resolver problemas de clasificación arbitrarios mientras se garantiza control tanto sobre la precisión del clasificador final como sobre su equidad. A diferencia de otros enfoques existentes, la propuesta de esta investigación se centra en proporcionar una interfaz única para solucionar problemas de clasificación arbitrarios. El sistema utiliza una representación basada en atributos protegidos y expone un conjunto de métricas de equidad. Internamente se realiza una búsqueda de la mejor configuración para ensamblar un grupo de arquitecturas de aprendizaje automático. Ambas características posibilitan obtener una solución agnóstica al problema a resolver. Su aplicabilidad a problemas de procesamiento de lenguaje natural potencia la relevancia del trabajo. El resultado final es una propuesta teórica, respaldada por un prototipo computacional, que demuestra que el estudiante posee las habilidades necesarias para aplicar en la práctica sus conocimientos.

Durante el desarrollo de esta investigación, Jorge ha tenido que asimilar por su cuenta conocimientos de diversas áreas, como optimización e inteligencia artificial. Además, ha tenido que estudiar en profundidad un campo de investigación tan novedoso y variado como es el análisis de sesgos en algoritmos de aprendizaje automático. El proceso de investigación e implementación desarrollado por Jorge queda recogido en un documento de tesis que avala además sus habilidades para llevar a buen término una investigación científica con la formalidad que el campo requiere. El estudiante ha demostrado así no solo dominio técnico del área, sino además capacidad de organización.  Todo esto lo han realizado a la par de las actividades docentes, como estudiante de pregrado y como alumno ayudante de la asignatura Programación, donde ha sabido asumir con éxito todas las responsabilidades y retos.

Conozco a Jorge desde que cursaba el primer año de la carrera, donde le impartí clases de la asignatura Programación. Habría sido fácil predecir desde aquel entonces que terminaría supervisando su tesis, no necesariamente porque él hubiera querido, sino porque yo no habría permitido que fuese de otra forma. Querer llevar sus conocimientos al límite es una característica que no se puede dejar pasar desapercibida en un estudiante, y desde el minuto uno estaba claro que Jorge la poseía. El querer aprender más, entender mejor y encontrar una forma de aplicarlo luego, forma parte de la naturaleza de Jorge. Desde aquel entonces he podido compartir con él como parte del colectivo de la asignatura, trabajando en proyectos del grupo de investigación, y participando en eventos a nivel de universidad, nacional, e incluso internacional. Son pocos los estudiantes que pueden afirmar haber tenido una trayectoria tan rica a lo largo de sus años de pregrado como la de Jorge. Por supuesto, no todo es color de rosa, pues si hay otra cosa que describe a Jorge es el estar probando cada dos por tres alguna tecnología nueva; lo cual es perfecto mientras quede solo de su lado, pero para sorpresa de nadie, siempre terminaba arrastrado también yo a probar sus nuevos juguetes; prueba de ello es que hoy en día me encuentro atrapado usando Windows de nuevo por su culpa, y lo peor es que solo, porque Jorge ya cambió nuevamente. Habiendo dicho esto, me siento muy complacido de haber podido trabajar con él. Espero en los próximos años podamos seguir compartiendo y brindándole mi apoyo en las nuevas experiencias que se avecinan, con las que estoy seguro crecerá aún más. Tengo plena confianza en que ha de cosechar las recompensas por todo el empeño que ha puesto en sus estudios y en la investigación, y que ejercerá como un excelente profesional.


\vspace{1cm}

\begin{flushright}
\emph{MSc. Juan Pablo Consuegra Ayala}\\
    Facultad de Matemática y Computación \\
    Universidad de La Habana    
\end{flushright}

\end{opinion}