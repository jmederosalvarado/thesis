\chapter{Análisis Experimental}\label{chapter:experiments}

En este capitulo se evalúa la capacidad de nuestro sistema de optimización de dos fases de resolver un problema de clasificación y lograr buenos resultados en métricas de precisión y equidad a partir de ensamblar modelos base. La experimentación realizada consiste de dos etapas, en una se analiza la capacidad de la primera fase del sistema de obtener un conjunto de modelos base lo suficientemente diverso como para que ensamblar estos resultados en un modelo de mayor precisión que los modelos base. En la segunda etapa, se estudia si el algoritmo propuesto permite encontrar formas de ensamblar los modelos base resultantes de la primera etapa de forma tal que se obtengan valores satisfactorios tanto de precisión como en las métricas de equidad.

El sistema descrito ene este trabajo fue evaluado en corpus correspondiente a la tarea \emph{HAHA 2019}(\textit{Humor Analysis based on Human Annotation}), con marco en el evento \emph{IberLEF 2019}. El corpus contiene 30000 tweets clasificados manualmente en idioma Español, de los cuales 24000 son utilizados para entrenamiento y 6000 para evaluación. Cada tweet es clasificado como humoroso o no. Para los experimentos que se reportan a continuación, solamente el conjunto de entrenamiento es utilizado apara ajustar el sistema.

% %%
% Experimentos:
%
% Los datasets van todos juntos
%
% Primera etapa:
%
% Capacidad del sistema de generalizar/ generar soluciones que generilicen
%
% Influencia de las metricas de diversidad en esto
%
% Encontrar la mejor combinacion de hiper que garantiza improvement
%
% Segunda etapa:
%
% Capacidad  del sistema para generar modelos justos con una sola fairness
%
% Multiples fairness a la vez
