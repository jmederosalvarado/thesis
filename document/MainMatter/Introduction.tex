\chapter*{Introducción}\label{chapter:introduction}
\addcontentsline{toc}{chapter}{Introducción}

En la actualidad los algoritmos de aprendizaje automático están siendo aplicados en disimilies áreas de la vida humana. Es común encontrarlos aplicados en sistemas de recomendacion de compras, aplicaciones de citas, solicitudes de préstamos, contratación personal y muchas otras áreas.  A raíz de ello, ha surgido un creciente interés en estudiar las potencialidades y limitaciones de los modelos de aprendizaje automático, así como las posibles implicaciones de confiar ciegamente en sus predicciones.

En particular, su incorporación a tareas de toma de decisiones de alto riesgo ha dirigido la atención de muchos investigadores hacia una nueva interrogante: ¿estarán siendo "justos" los algoritmos de aprendizaje automático al tomar sus decisiones?

En este escenario, ha ganado popularidad el desarrollo de técnicas para detectar y mitigar los sesgos en colecciones de datos y algoritmos de aprendizaje automático. Tales herramientas son cruciales para desarrollar sistemas de toma de decisiones mas justos. Los estudios orientados hacia la equidad en algoritmos de aprendizaje automático se enfocan principalmente en desarrollar técnicas que consideren tanto la precisión como la equidad de los modelos.

\section*{Motivación}

Un modelo de aprendizaje de máquina se entrena con el objetivo de optimizar una única métrica, en la mayoría de los casos la precisión. Esto significa que los modelos aprenden muy bien los patrones que se presentan en los datos de entrenamiento, incluyendo aquellos patrones que representan sesgos y prejuicios que están desafortunadamente presente en la sociedad y por ende en los datos recopilados, en algunos casos incluso amplifican estos patrones negativos. Son varias la técnicas que se han explorado para resolver este problema, algunas se enfocan en un preprocesamiento de los datos para eliminar aquellos elementos que puedan inducir un sesgo en el modelo, otras realizan variacinoes en el metodo de entrenamiento con el mismo objetivo. Sin embargo permanece relativamente poco explorado el uso de técnicas de optimizacion multiobjetivo que permitan al modelo optimizar hasta encontrar un buen balance entre cuan justo es y cuan preciso.

Otra tecnica que ha demostrado ser de gran utilidad en la prevencion de los sesgos en los modelos de aprendizaje de máquina es la construccion de ensamblados de multiples modelos que maximizan la varianza entre si, por lo que se minimiza el sesgo del ensamblado final.

\section*{Hipótesis}

Se espera que la utilización de un enfoque multiobjetivo sea clave para encontrar modelos que a la vez que son precisos y tienen gran desempeño en la tarea en cuestión, son también modelos justos de acuerdo no solo a una sino a varias metricas de equidad.

\section*{Preguntas Cientificas}

\begin{itemize}
    \item Establecer si es posible aplicar de alguna forma uno de los algoritmos de optimización multiobjetivo en la litaratura en el contexto de el aprendizaje de máquina automático en un espacio de búsqueda heterogéneo.
    \item Establecer si estos algoritmos de optimización multiobjetivo son verdaderamente efectivos para encontrar modelos justos.
\end{itemize}

\section*{Problematica}

A pesar de que existe AutoGOAL, una biblioteca de AutoML, que permite obtener modelos para resolver problemas arbitrarios utilizando entre otras técnicas aprendizaje de máquina. No existe una biblioteca o herramienta que permita resolver de principio a fin un problema de clasificacion utilizando aprendizaje de máquina y donde exita alguna garantia de que el modelo aprendido sea justo.

\section*{Objetivo general}

Proponer una herramienta que permita resolver problemas de clasificacion utilizando aprendizaje de máquina y que permita garantizar que el modelo aprendido sea justo.

\section*{Objetivos especifico}

\begin{itemize}
    \item Encontrar modelos que maximicen la varianza para minimizar el sesgo.
    \item Metodos basados en metaheuristicas para optimizar los modelos utilizando simultaneamente metricas de equidad y precisión.
    \item Explorar adicion de optimizacion multiobjetivo a AutoGOAL para que el modelo aprendido sea justo.
    \item Metodos basados en la combinacion de diferentes metricas en una sola, para poder aprovechar los multiples metodos de optimizacion que existen.
\end{itemize}
