\chapter*{Introducción}\label{chapter:introduction}
\addcontentsline{toc}{chapter}{Introducción}

% motivación + problemática + objetivos + contribuciones + párrafo con la estructura del proyecto

%(1) que ML está en todos lados

En los últimos años, ha habido un creciente interés en aplicar técnicas de aprendizaje automático para resolver diferentes tipos de tareas.
Análisis de emociones en textos, etiquetado de imágenes y traducción automática son algunos ejemplos de estas tareas~\parencite{bahdanau2014neural,ucidata,mnih2013machine,yavuz2022comparison}.
En general, los problemas de clasificación son uno de los problemas más comunes para los cuales los algoritmos de aprendizaje automático tienden a producir buenos resultados.

%(2) que aplicarlo a toma de decisiones de alto riesgo da lugar a el tema de los sesgos

Con el uso cada vez más frecuente de los métodos de aprendizaje automático en dominios tales como \emph{prestamos financieros}, \emph{contratación}, \emph{justicia criminal}, y \emph{admisiones en las universidades}, ha habido una mayor preocupación por el potencial de estas técnicas de accidentalmente codificar sesgos sociales y resultar en una discriminación sistemática~\parencite{prorepublica2016bias,barocas-hardt-narayanan,NIPS2016_a486cd07,Buolamwini2018GenderSI,caliskan2017semantics}.
Un clasificador que solamente es ajustado para maximizar la precisión de las predicciones puede injustamente predecir un alto riesgo en el crédito para algunos subgrupos de la población aplicando a un préstamo.
Un ejemplo de tratamiento imparcial a grupos de la población puede ser encontrado en el \emph{software} \emph{COMPAS} (\emph{Correctional Offender Management Profiling for Alternative Sanctions}), utilizado por las cortes en los Estados Unidos para determinar los riesgos de un individuo de reincidir en un crimen.
Los resultados que produce este sistema son utilizados para motivar decisiones respecto a si los defendidos deben ser liberados, en diferentes etapas del proceso de justicia.
Problemáticamente, este software falsamente etiquetaba defendidos de raza no blanca con un mayor riesgo que los defendidos blancos~\parencite{prorepublica2016bias}.

%(3) que si muchos approach existentes para mitigar son de optimización para alcanzar un trade off

Cuantificar y mitigar los sesgos durante las diferentes etapas del ciclo de vida de los modelos de aprendizaje automático ha sido objeto de estudio de numerosas investigaciones~\parencite{barocas-hardt-narayanan}.
Específicamente, un número de métodos han sido propuestos recientemente para incrementar la equidad en los resultados producidos por estos modelos.
La clave del diseño detrás de todos estos métodos es maximizar la precisión de las predicciones, sujeto a restricciones de equidad sobre los resultados.
Sin embargo, como consecuencia de mantener la tratabilidad computacional de estas técnicas, estos métodos sufren de una o más de las siguientes desventajas.
La técnica de mitigación es
(i) específica a la clase del modelo utilizado (p.e. solamente modelos lineales),
(ii) limitada a un conjunto específico de definiciones de equidad,
(iii) limitado a un único atributo protegido binario,
(iv) requiere acceso a información sensible en el momento de las predicciones, o
(v) resulta en un clasificador \emph{randomizado} que puede generar diferentes predicciones para la misma entrada en diferentes momentos.
Estas desventajas limitan la habilidad de los profesionales de poner en funcionamiento modelos justos de aprendizaje automático para problemas de la práctica con objetivos y restricciones arbitrarias.
Por ejemplo, como resultado de la limitación (i), cualquier ensemble o solución de aprendizaje híbrido combinando diferentes clases de modelos y/o conocimiento del dominio son descartadas.
Una consecuencia de (iv) es que se necesita información acerca de atributos sensibles en el momento de realizar las predicciones, lo cual puede traer problemas con las leyes que protegen la privacidad de los usuarios.
Finalmente, (v) puede resultar en el mismo modelo, aceptando y denegando préstamos para el mismo individuo en diferentes ocasiones de forma arbitraria.

%(4) que si AutoML es bueno para el tema de democratizar una tecnología y que ayuda a crear hipótesis automáticamente

Recientemente, avances en \emph{AutoML} (del inglés \emph{Automatic Machine Learning}) han permitido el desarrollo de bibliotecas y herramientas efectivas para encontrar la mejor combinación de algoritmos e hiperparametros para resolver un problema.
Múltiples tecnologías han sido propuestas para resolver el problema de \emph{AutoML}, tales como \emph{AutoWeka}~\parencite{autoweka} o \emph{AutoKeras}~\parencite{autoKeras}.
Estas herramientas son alternativas para reducir el tiempo empleado por investigadores resolviendo problemas bien estudiados.
Incluso si las herramientas de \emph{AutoML} tienden a consumir más tiempo y recursos que bibliotecas estándar de aprendizaje automático, no tener que razonar acerca de cual arquitectura podría ser la más apropiada para el problema en cuestión vale el esfuerzo de desarrollarlas.
Además, al ser aplicables a un rango amplio de problemas, aprender a utilizar estas herramientas puede ser más sencillo que aprender a utilizar diferentes bibliotecas independientes de aprendizaje automático.
Se podría decir entonces que uno de los objetivos de las técnicas de \emph{AutoML} es la \emph{democratización del Aprendizaje Automático}.

%(5) que si AutoGoal tiene varias ventajas sobre el resto

En este contexto se encuentra también \emph{AutoGOAL}~\parencite{autogoal} como un buen ejemplo de estas bibliotecas de \emph{AutoML}.
\emph{AutoGOAL} es una propuesta reciente que utiliza técnicas que le permite explorar un espacio de búsqueda heterogéneo de modelos e hiperparámetros.
A diferencia de otras tecnologías de \emph{AutoML} existentes, \emph{AutoGOAL} puede de forma automática construir flujos que le permitan combinar técnicas y algoritmos de diferentes bibliotecas, incluyendo clasificadores lineales, redes neuronales y herramientas de procesamiento del lenguaje.
A pesar de que \emph{AutoGOAL} es capaz de combinar algoritmos de diferentes categorías para construir flujos, no tiene la habilidad de combinar múltiples flujos para generar una solución.
Adicionalmente, su algoritmo de búsqueda le permite solamente optimizar una función objetivo y no permite atacar problemas que en su naturaleza son de múltiples objetivos.

%(6) finalmente que a partir de combinar ciertas cosas se pueden resolver las limitaciones de otras propuestas actuales

Las técnicas de \emph{AutoML} y en particular la estrategia que propone \emph{AutoGOAL} resulta sumamente útil para superar las limitación que tienen muchos de los enfoques de mitigación de sesgos al solo ser aplicables a determinadas clases de modelos y métricas.
La idea es utilizar la capacidad \emph{AutoGOAL} de explorar un espacio de modelos heterogéneo y componer diferentes técnicas de aprendizaje automático, para poder encontrar una solución que cumpla con determinadas restricciones de equidad.
Sin embargo, se mantiene la interrogante de como lograr mantener determinadas restricciones de equidad a la vez que el método se mantiene agnóstico al modelo de aprendizaje.
Para ello resultaría muy útil poder utilizar \emph{AutoGOAL} en entornos multiobjetivo, donde se optimizan simultáneamente métricas de equidad y de efectividad del modelo, esto es precisamente lo que permite lograr una de las propuestas de este trabajo.
Adicionalmente, se propone una solución al problema de \emph{AutoGOAL} de no poder combinar diferentes flujos de extremo a extremo en un flujo compuesto.
Esto se logra a partir de la utilización de métodos de ensemble, los cuales permiten la combinación de las predicciones de diferentes modelos base para lograr modelos finales más robustos~\parencite{polikar2006ensemble}.

%De ahí caes en objetivos, contribuciones y organización

\section*{Objetivo general}

El objetivo de este trabajo es proponer y estudiar un sistema que permita la solución de problemas de clasificación arbitrarios de forma justa.
Este sistema debe ser agnóstico al modelo de solución y el proceso de optimización del mismo, así como a las métricas de equidad y efectividad que se deseen utilizar.
De esta forma contribuir a la democratización de las técnicas de solución de problemas de clasificación de forma justa.

\section*{Objetivos específicos}

\begin{itemize}
    \item Realizar un estudio de la literatura en los temas relacionados a la mitigación de sesgos y otros temas relevantes a la solución propuesta.
    \item Concebir un método que permita atacar el problema de encontrar clasificadores justos y efectivos para problemas de clasificación arbitrarios de forma agnóstica a los modelos utilizados.
    \item Implementar un prototipo computacional del método propuesto.
    \item Diseñar un marco experimental donde evaluar los resultados del método de solución diseñado.
    \item Comparar los resultados del sistema propuesto con otros métodos propuestos en la literatura.
    \item Analizar en profundidad los resultados obtenidos y arribar a conclusiones.
\end{itemize}

%This paper’s objective is to design and validate a system that takes advantage of all the different architectures that are
%generated while solving classification problems with Auto-ML tools, particularly AutoGOAL, to produce more robust classifiers.
%We propose a two-phase optimization system based on Automatic Machine Learning (Auto-ML) for solving classification
%problems. In the first phase, the system follows a probabilistic strategy to find the best combination of algorithms and
%hyperparameters to generate a collection of base models according to certain diversity criteria; and in the second, it follows
%similar Auto-ML strategies to ensemble those models.
%The specific contributions of this research are as follows:
%  We propose a system based on AutoGOAL and ensemble methods to solve arbitrary classification problems. The improvement
%in the generalization capability of the solutions is the main advantage of our proposal, which is available online as a
%python project for the research community.
%  We study how some diversity measures can impact the performance of an ensemble method when used to ensure diversity
%between the collection of base classifiers.
%  We provide the basis for future studies on unfairness and bias mitigation. Since the optimization process is divided into
%two phases, a second loss function can be optimized after the first phase by imposing constraints to limit detriment in
%performance.
%The remainder of the paper is organized as follows. Section 2 gives a brief overview of current research in Auto-ML techniques
%and ensemble methods. Section 3 presents our two-phase optimization system for solving classification problems.
%Section 4 shows the results of evaluating the system. Section 5 discusses the implications of the results we obtained from
%the experiments. Finally, Sections 6 presents the conclusions, along with future lines of work.

% importante vender todas las contribuciones explícitamente (eran como 4, y las anotaste)

\section*{Contribuciones}

\begin{itemize}
    \item Se propone un sistema, cuya implementación queda disponible a la comunidad, que permite resolver problemas de clasificación arbitrarios con control sobre la equidad.
    \item Este sistema permite al usuario especificar diferentes funciones de equidad a optimizar.
    \item Adicionalmente, trabaja sobre colecciones de datos heterogéneas.
    \item El sistema es agnóstico al modelo y no requiere conocimiento específico del dominio del problema para utilizarlo satisfactoriamente.
\end{itemize}

\section*{Organización del trabajo}

El resto de este trabajo esta organizado de la siguiente forma.
El capítulo~\ref{chapter:state-of-the-art} realiza una revisión de la literatura y el estado del arte en los tópicos relacionados con el problema a resolver.
Luego el capítulo~\ref{chapter:proposal} describe detalladamente el método de solución propuesto.
El capítulo~\ref{chapter:experiments} describe la experimentación realizada, los resultados obtenidos y un análisis en profundidad de los mismos.
Finalmente se arriban a conclusiones y se discuten las líneas de investigación futuras.
